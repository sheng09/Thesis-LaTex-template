
% ********************************************************************
% Useful commands
% ********************************************************************
\newcommand{\ie}{i.\,e.~}
\newcommand{\Ie}{I.\,e.~}
\newcommand{\eg}{e.\,g.~}
\newcommand{\Eg}{E.\,g.~}

%----------------------------------------------------------------------------------------

% Define some commands to keep the formatting separated from the content 
\newcommand{\keyword}[1]{\textbf{#1}}
\newcommand{\tabhead}[1]{\textbf{#1}}
\newcommand{\code}[1]{\texttt{#1}}
\newcommand{\file}[1]{\texttt{\bfseries#1}}
\newcommand{\option}[1]{\texttt{\itshape#1}}
%----------------------------------------------------------------------------------------

%----------------------------------------------------------------------------------------
%	Optional Packages
%----------------------------------------------------------------------------------------
\usepackage{graphicx}
\usepackage[figuresright]{rotating}
\usepackage[utf8]{inputenc} % Required for inputting international characters
\usepackage[T1]{fontenc} % Output font encoding for international characters
%\usepackage{mathpazo} % Use the Palatino font by default
\usepackage{mathptmx} % Sheng prefer font close to New Roman
%\usepackage{tgpagella} %The TEX Gyre Pagella family of fonts is based on the URW Palladio family, but heavily extended (https://www.tug.org/FontCatalogue/texgyrepagella/)

\usepackage{fontawesome} % Use for loading symbol for URLs
\usepackage{textcomp} % use for degree symbol (\textdegree)
\usepackage{pdfpages}
%\usepackage[document]{ragged2e}

\usepackage{pdftexcmds}
\usepackage[section]{minted} %works well in Overleaf but offline may need Python package to be installed and shell escape setting applied
\usemintedstyle{friendly}
\usemintedstyle{borland}

\usepackage[autostyle=true]{csquotes} % Required to generate language-dependent quotes in the bibliography
\usepackage[font={itshape,raggedright},begintext=``,endtext="]{quoting}
\usepackage{microtype}
\usepackage[framemethod=TikZ]{mdframed}
\usepackage{amssymb}
\usepackage{multirow}
\usepackage{tabulary}
\usepackage{glossaries}
\usepackage{textcomp}

% Sheng
\DeclareUnicodeCharacter{1EA1}{\d{a}} % Sheng
\newcommand{\SORTCITATION}[1]{}
%The following allow display of equations with better presentation -----
\usepackage[retainorgcmds]{IEEEtrantools}
\usepackage[titles]{tocloft}

\newcommand{\listequationsname}{List of Equations}
\newlistof{myequations}{equ}{\listequationsname}
\newcommand{\myequations}[1]{%
\addcontentsline{equ}{myequations}{\protect\numberline{\theequation}#1}\par}
\setlength{\cftmyequationsnumwidth}{2.5em}% Width of equation number in List of Equations

\usepackage{amsmath,bm}

\newcommand{\minus}{\scalebox{0.75}[1.0]{$-$}}
% End of Sheng's revisions

%New colors defined below
\definecolor{LightGray}{gray}{0.9}
\definecolor{codegreen}{rgb}{0,0.6,0}
\definecolor{codegray}{rgb}{0.5,0.5,0.5}
\definecolor{codepurple}{rgb}{0.58,0,0.82}
\definecolor{backcolour}{rgb}{0.95,0.95,0.92}
\definecolor{mintedbackground}{rgb}{0.95,0.95,0.95}
\definecolor{revblue}{rgb}{0.15,0.48,1.0}
\definecolor{revlightblue}{rgb}{0.73,0.83, 0.99}
\definecolor{revred}{rgb}{0.82,0.20,0.22}
\definecolor{revlightred}{rgb}{0.99,0.76,0.77}

%\usepackage{subcaption} %for sub figures

\usepackage{fixme} %FixMe package
\fxsetup{status=draft,author=} % <====== add this line
\fxsetup{theme=color}
\fxsetface{margin}{\scriptsize}
%\definecolor{fxnote}{rgb}{0.0000,0.0000,0.0000}
%\definecolor{fxwarning}{rgb}{0.0000,0.0000,0.0000}
%\definecolor{fxerror}{rgb}{0.0000,0.0000,0.0000}
%\definecolor{fxfatal}{rgb}{0.0000,0.0000,0.0000}

\pdfcompresslevel=9 %best compression
\pdfadjustspacing=1 %for small font expansion

%\usepackage{chngcntr}
%\counterwithin{figure}{section}

%\usepackage[titles]{tocloft}
%\setlength{\cftbeforechapskip}{5pt} \ sets spacing in Table of Contents
% or see http://www-h.eng.cam.ac.uk/help/tpl/textprocessing/squeeze.html

%----------------------------------------------------------------------------------------
%	BOX SETTINGS
%----------------------------------------------------------------------------------------
% from https://texblog.org/2015/09/30/fancy-boxes-for-theorem-lemma-and-proof-with-mdframed/

%Proof
\newcounter{prf}[section]\setcounter{prf}{0}
\renewcommand{\theprf}{\arabic{chapter}.\arabic{section}.\arabic{prf}}
\newenvironment{prf}[2][]{%
\refstepcounter{prf}%
\ifstrempty{#1}%
{\mdfsetup{%
frametitle={%
\tikz[baseline=(current bounding box.east),outer sep=0pt]
\node[anchor=east,rectangle,fill=red!20]
{\strut Proof~\theprf};}}
}%
{\mdfsetup{%
frametitle={%
\tikz[baseline=(current bounding box.east),outer sep=0pt]
\node[anchor=east,rectangle,fill=red!20]
{\strut Box~\theprf:~#1};}}%
}%
\mdfsetup{innertopmargin=10pt,linecolor=red!20,%
linewidth=2pt,topline=true,%
frametitleaboveskip=\dimexpr-\ht\strutbox\relax
}
\begin{mdframed}[]\relax%
\label{#2}}{\end{mdframed}}
%%%%%%%%%%%%%%%%%%%%%%%%%%%%%%


%----------------------------------------------------------------------------------------
%	MARGIN SETTINGS
%----------------------------------------------------------------------------------------

\geometry{
	paper=a4paper, % Change to letterpaper for US letter
	inner=2.5cm, % Inner margin
	outer=2.5cm, % Outer margin
	bindingoffset=.5cm, % Binding offset
	top=1.5cm, % Top margin
	bottom=1.5cm, % Bottom margin
	%showframe, % Uncomment to show how the type block is set on the page
}


%----------------------------------------------------------------------------------------
%	Commands for revising the thesis
%   by Sheng Wang
%----------------------------------------------------------------------------------------
\usepackage{soul}

% command to add
\newcommand{\revadd}[1]{{\color{revred} {#1}}}
%\newcommand{\revadd}[1]{#1}

% command to delete something
\newcommand{\revdel}[1]{{\color{revred} \st{#1}}}
%\newcommand{\revdel}[1]{}


% command to add margin comments
\usepackage[textwidth=1.7cm]{todonotes}
\setlength{\marginparwidth}{2.1cm}
\newcommand{\revcom}[1]{\todo[bordercolor=revlightred,linecolor=revlightred,backgroundcolor=revlightred,size=\scriptsize]{#1}}
%\newcommand{\revcom}[1]{}

% linenumbers
\usepackage{lineno}
\renewcommand\linenumberfont{\normalfont\tiny}
\newcommand{\setlinenumbersheng}{\linenumbers}
%\newcommand{\setlinenumbersheng}{}