\chapter{Introduction}\label{ch:intro}

\section{Background and Motivation}\label{sec:intro_background}


Global seismology has come a long way in exploring and understanding the structure and dynamics of the Earth's interior. Powered by numerous forward and inverse techniques, seismological studies on a global scale established a one-dimensional radial structure of the Earth's interior that featured discontinuities and distinct spherical shells: the crust, mantle, outer core (OC), inner core (IC), and most recently, the innermost inner core \citep{dziewonski_preliminary_1981,kennett_constraints_1995,ishii_innermost_2002}. These shells are evidence of the Earth's and planetary differentiation processes \citep{carlson_mechanisms_1994,walter_early_2004}. \revadd{For example, the core existence reveals the segregation process of an iron-rich core from a silicate mantle which is one of the most formative events that sets the conditions for the entire evolution that follows \citep{olson_core_2022,wood_accretion_2006}. Also, the core plays an active role in fundamental processes. On the Earth, the coupling between its OC and IC through the transfer of material and heat actively affects the generation and variations of the geomagnetic field \citep[e.g.,][]{braginsky_structure_1963,buffett_thermal_1996,hollerbach_influence_1993,roberts_geomagnetic_2008}, dynamics of the lowermost mantle \citep[e.g.,][]{aubert_thermochemical_2008,gubbins_melting_2011}, and even processes at Earth's free surface \citep[e.g.,][]{biggin_palaeomagnetic_2015}. On Mars, a relatively large Martian core enriched with light elements, detected by seismic observations, could explain why the red planet lost its magnetic field \citep[e.g.,][]{stahler_seismic_2021,yokoo_stratification_2022}.} In more recent times, three-dimensional models of the Earth \citep{dziewonski_mapping_1984,van_der_hilst_evidence_1997} and their further refinements helped confirm and establish a whole range of structures, such as the mantle transition zone \citep{ritsema_global_2004}, the subducted slabs \citep{widiyantoro_structure_1996}, and large-scale heterogeneities at the base of the mantle \citep{garnero_structure_2008}.\revcom{Revised in response to comment \#3 of of The Associate Dean and the Delegated Authority, and comments \#2 and \#10 of Examiner \#3.}



In the last two decades, the discovery of long-range cross-correlations in the seismic wavefield has allowed for significant advancement in detailed imaging of Earth's subsurface \citep[e.g.,][]{campillo_long-range_2003,shapiro_high-resolution_2005}. In the ambient noise wavefield, the cross-correlation of continuous records between two seismic receivers resembles the direct seismic waves as if measured at one of the receivers from a virtual source at the other. This phenomenon is thoroughly analysed, and the term ``reconstruction of seismic waves'' is coined, which provides new scope for the recovery of subsurface architecture \citep[e.g.,][]{lobkis_emergence_2001,snieder_extracting_2004,wapenaar_tutorial_2010}. There are numerous successful practices of imaging subsurface structures via reconstructing surface wave propagations between receivers \citep[e.g.,][]{yao_surface-wave_2006,lin_ambient_2007,lin_surface_2008,moschetti_surface_2007,bensen_broadband_2008}.




Much attention has also been gained in global-scale cross-correlation studies and their geophysical inference on the Earth's internal structure. After adopting similar cross-correlation computation routines to \revadd{seismic records from worldwide networks, the obtained correlation stacks exhibit many prominent features similar to seismic body waves, especially deep-Earth phases \citep[e.g.,][]{lin_extracting_2013,boue_teleseismic_2013}. They were hypothesised to represent ``reconstructed body waves''.}\revdel{to the late coda records of globally distributed large earthquakes, the obtained coda correlations exhibit many prominent features similar to seismic body waves, especially deep-Earth phases.} \revadd{However, subsequent studies demonstrate that the emergence of the features strongly depends on the late coda of significant earthquakes, such as the records in 3-9 hr after the earthquake origin time \citep{lin_extracting_2013,boue_reverberations_2014}. The late coda mainly consists of deep-travelling reverberations (or high-quality-factor modes) with energy confined in great-circle planes \citep{maeda_constituents_2006,sens-schonfelder_lack_2015,poli_analysis_2017}. Thus, the primary condition for reconstructing seismic waves in which the wavefield should be equipartitioned or diffuse is not satisfied. Namely, the wavefield energy distribution should be equal in different directions \citep[e.g.,][]{lobkis_emergence_2001,campillo_long-range_2003}.} Moreover, some coda correlation features do not have equivalent phases in travel-time-distance stacks, and some are non-causal (arriving before P waves), prompting the community to refer to them as ``spurious''. Furthermore, some features present unstable timing and mismatch amplitude to corresponding body waves or abnormal polarisations, such as the ScS-like features in vertical-to-vertical cross-correlation stacks at near-zero distances \citep{boue_teleseismic_2013,boue_reverberations_2014}. \revdel{Moreover, the emergence of the features strongly depends on the late coda of major earthquakes. The late coda mainly consists of deep-travelling reverberations (or high quality-factor modes) with energy confined in great-circle planes other than being equipartitioned as necessary for accurate reconstructions of seismic waves.} \revcom{Revised in response to comments \#2 and \#3 of Examiner \#1, and comment \#2 of Examiner \#3.}


Recent studies revealed that all seismic event coda correlation features could be explained through a mathematical conjecture and simple laws of physics that employs ray theory: they arise due to the similarity between two body waves that share a portion of ray path in common from the same source to a pair of receivers 
%(Ph\d{a}m et al., 2018; Tkalčić and Ph\d{a}m 2018; Kennett and Ph\d{a}m 2018)
\citep{pham_earths_2018,tkalcic_shear_2018,kennett_nature_2018}. 
In other words, both causal (where the correlation features are like seismic body waves) and non-causal features (where the correlation feature do not correspond to any seismic waves) in correlograms are produced by interferences between seismic waves. For example, the ScS-like feature can be formed by the similarity of the following seismic phases and their time differences: SKSScS-SKS, ScSSKS-SKS, SKSSKSScS-SKSSKS, and there are nearly infinite wave pairs in the late coda time window.





For the above reasons, a new procedure is required to effectively utilise seismic event coda correlation features. It should be different from ambient-noise correlation practices that treat correlations as the reconstruction of seismic waves. The coda correlations as a new type of observations complementary to direct seismic wave observations would provide new information about the Earth's and planetary interiors through well-established global tomographic practices.



\section{Thesis Objectives and Outline}\label{sec:intro_objectives}

This study aims at a solution, an accurate procedure for using seismic event coda correlations, to pose new constraints on the Earth's internal structure on a global scale. To achieve this goal, this thesis consists of theoretical analyses and derivations, method developments, applications, and further extensions, as listed below.

In \textbf{Chapter} \ref{ch:theo} \citep{wang2020seismic}, we present comprehensive analyses to quantitatively `dissect' coda correlation features for their formation mechanism. Via separating, determining, and analysing individual `constituents' of a correlation feature, we provide observational evidence as direct proof to the mathematical conjecture that body-wave interferences form coda correlations. We also derive a quantitative relationship between correlation features and the Earth's internal structure. This provides a practical understanding and theoretical base for accurately utilising global correlations.

In \textbf{Chapter} \ref{ch:tomo} \citep{wang2020seismictomo}, we feature a novel framework toward global coda correlation tomography based on Chapter \ref{ch:theo}. We verify the new framework via experiments. We demonstrate that global coda correlations pose new constraints on the Earth's inner core structure. We compare the new approach with the method based on the assumption of seismic wave reconstructions. We illustrate that significant inaccuracy can arise in tomographic images if global correlation features are treated as reconstructed seismic waves. This paves the way for further detailed and application-oriented method improvements and exploitation of global correlation tomography.

\textbf{Chapter} \ref{ch:janiso} \citep{wang_shear-wave_2021} presents an application of global correlations to provide a new class of observations for inner-core shear-wave anisotropy within the new framework developed in Chapters \ref{ch:theo} and \ref{ch:tomo}. We find that inner-core shear waves travel faster by at least ~5 s in the direction oblique to the Earth's rotation axis than the direction parallel to the equatorial plane. Among multiple models, the simplest explanation is the inner-core shear-wave cylindrical anisotropy with a minimum strength of ~0.8\%, formed through the lattice-preferred-orientation mechanism of iron. That places new constraints on the inner core mineral composition. Although we cannot uniquely determine its stable iron phase, the new observations rule out one of body-centred-cubic iron models.

\textbf{Chapter} \ref{ch:review} \citep*{tkalvcic2022shear} presents a review of the shear properties of the Earth's inner core from body-wave, normal-mode and our new global coda-correlation studies. The shear waves that provide direct evidence for the inner core solidity remained elusive and were reported in only a few publications 
%(Cao et al., 2005; Deuss et al., 2000; Julian et al., 1972; Okal and Cansi, 1998; Wookey and Helffrich, 2008)
\citep{julian_pkjkp_1972,okal_detection_1998,deuss_observation_2000,cao_observation_2005,wookey_inner-core_2008}, likely because of the very weak amplitude of shear waves 
%(Shearer et al., 2011)
\citep{shearer_visibility_2011}. We reviewed historical and contemporary studies for the shear properties of the inner core for its (1) solidity and shear-wave speed, (2) attenuation in shear, (3) shear-wave anisotropy, and presented (4) challenges and new research directions, including new observations of PKJKP waves. As a new paradigm, the global coda correlations hold the keys to refined inner-core shear properties' measurements, informing dynamical models and strengthening interpretations of the inner-core anisotropic structure and viscosity.

In \textbf{Chapter} \ref{ch:intersrc} \citep{wang2022inter-src}, we extend the method to cross-correlations between globally distributed source events, inter-source correlations, and devise a new procedure for scanning the planetary interiors with global inter-source correlations via merely a single seismograph. We first demystify how to build global inter-source correlations, which provides an answer and solution to the issue that conventional attempts fail to present the theoretically expected global inter-source correlation features. We then show that a single seismic station is sufficient to produce global correlation features sensitive to the Earth's and internal planetary structures. We utilise single-station correlograms to demonstrate implementations to constrain the sizes of Earth's and Martian cores, and we confirm a large Martian core. This provides a new paradigm for elucidating planetary interiors with currently realisable resources.

\textbf{Chapter} \ref{ch:summary} concludes this thesis with outlooks for future works.

